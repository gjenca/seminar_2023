\documentclass{article}
% vim: set indentexpr=
\def\documentlanguage{english}
\usepackage[all,pdf,2cell]{xy}\UseAllTwocells\SilentMatrices
\usepackage{amsthm}
\usepackage{amsmath}
\usepackage{amsfonts}
\newtheorem{theorem}{Theorem}[section]
\newtheorem{lemma}[theorem]{Lemma}
\newtheorem{proposition}[theorem]{Proposition}
\theoremstyle{definition}
\newtheorem{definition}[theorem]{Definition}
\newtheorem{example}[theorem]{Example}
\newcommand{\newcategory}[1]{\expandafter\newcommand\csname #1\endcsname{\mathbf{#1}}}
\newcategory{RelPosInv}
\newcategory{RelPos}
\newcategory{PosInv}
\newcategory{Pos}
\newcategory{Sup}
\begin{document}
\title{What we know about non-quantum things`}
\section{The categories}
\begin{itemize}
\item $\Pos$ is the category of posets and monotone maps.
\item $\RelPos$ is the category of posets and monotone relations.
\item $\PosInv$ is the category of posets equipped with an antitone involution and involution preserving maps.
\item $\RelPosInv$ is the category of posets equipped with an antitone involution and monotone relations. 
\end{itemize}

\section{$\Pos$}

\begin{itemize}
\item $\Pos$ is cartesian closed.
\item There is a functor $~^{op}\colon\Pos\to\Pos$ that takes a poset to an opposite poset.
\item There is a monad $D\colon\Pos\to\Pos$ that takes every poset to its poset of downsets ordered by inclusion.
\item $\Pos^D\simeq\Sup$ -- the Eilenberg-Moore category.
\end{itemize}

\section{$\RelPos$}

\begin{itemize}
\item $\RelPos$ is a compact category.
\item The dual object is the dual poset, which is in this context denoted by $~^*$.
\item $\RelPos$ is $\Pos$-enriched.
\item $\RelPos\simeq\Pos_D$ -- the Kleisli category.
\end{itemize}

\section{$\PosInv$}
\begin{itemize}
\item The objects of $\PosInv$ can be represented as isomorphisms $'\colon P\to P^{op}$.
\item The morphisms are then commutative squares.
\item $\PosInv$ is cartesian closed.\footnote{What was the exponential object? I forgot.}
\item The forgetful functor $U\colon\Pos\to\PosInv$ is both left and right adjoint.
\item The left adjoint to $U$ is $P\mapsto D(P)\oplus(D(P))^{op}$, where $\oplus$ is the coproduct in $\Pos$.
\item The right adjoint to $U$ is $P\mapsto D(P)\times(D(P))^{op}$.
\end{itemize}

\section{$\RelPosInv$}
\begin{itemize}
\item The objects may be represented as $\RelPos$-isomorphisms $P\to P^*$, morphisms remain the same.
\item The contravariance of $~^*$ then gives us a dagger: if $'\colon P\to P^*$, $'\colon Q\to Q^*$ are 
objects of $\RelPosInv$ and $f\colon P\to Q$ is a morphism (a monotone relation), then
$f^\dag\colon Q\to P$ is just $(')^{-1}\circ f^* \circ '$.
\item This is a dagger-compact category.
\item Effect algebras are certain dagger-Frobenius algebras in $\RelPosInv$.
\footnote{This was the original motivation to look at $\RelPosInv$}.
\item There is a chain of adjunctions $\RelPos\leftrightarrow\Pos\leftrightarrow\PosInv$,
inducing a monad $T$ given by $T(P)=D(P)\times (D(P))^{op}$ on $\PosInv$.
\item $\PosInv_T\simeq\RelPosInv$
\item $\PosInv^T\simeq\Sup$

\end{itemize}
\end{document}

