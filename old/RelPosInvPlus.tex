\documentclass{article}
\def\documentlanguage{english}
\usepackage[all,pdf,2cell]{xy}\UseAllTwocells\SilentMatrices
\usepackage{amsthm}
\usepackage{amsmath}
\usepackage{amsfonts}
\newtheorem{theorem}{Theorem}[section]
\newtheorem{lemma}[theorem]{Lemma}
\newtheorem{proposition}[theorem]{Proposition}
\theoremstyle{definition}
\newtheorem{definition}[theorem]{Definition}
\newtheorem{example}[theorem]{Example}
\newcommand{\newcategory}[1]{\expandafter\newcommand\csname #1\endcsname{\mathbf{#1}}}
\newcategory{PosInv}
\newcategory{SupInv}
\begin{document}

\section{The objects}

An involution on a poset $P$ is an antitone selfmap $x\mapsto x'$
such that $x''=x$. A poset with involution is called an {\em involutive poset}.

Let $P$ be an involutive poset.
\begin{itemize}
\item We write $D(P)$ for the set of all order ideals/downsets/lower sets of $P$.
\item The orthogonality relation $\perp$ on $P$ is given by the rule
	$$x\perp y\Leftrightarrow x\leq y'$$
Note that $\perp$ is symmetric.
\item For a subset $X$ of $P$, we write $X'=\{x':x\in X\}$.
\item For a subset $X$ of $P$ and $y\in P$, we write $y\leq X$ to express the fact that $y$ is a lower bound of $X$; similarly for upped bounds.
\item The set of all lower bounds of $X$ is denoted by $X^\downarrow$, $X^\uparrow$
	are all upper bounds of $X$.
\item The \emph{orthoclosure} of a subset $X$ of $P$ is the set
	$$X^\perp=\{y\in P:(\forall x\in X)x\perp y\}.$$
	Note that $X^\perp=X'^\downarrow$. In particular, $X^\perp$ is a lower set of $P$.
\item $I\mapsto I^{\perp\perp}$ is a closure operator on $D(P)$.
\item A lower set $I$ of $P$ is said to be \emph{orthoclosed} iff $I=I^{\perp\perp}$.
\end{itemize}

\begin{lemma}
Let $P$ be an involutive poset. For every $I\in D(P)$ 
$I^\uparrow=(I^\perp)'$
\end{lemma}
\begin{proof}
$x\in I^\uparrow$ iff $x\geq I$ iff $x'\leq I'$ iff $x'\in (I')^\downarrow=I^\perp$ iff $x\in (I^\perp)'$.
\end{proof}

\section{The adjunction}

The categories are
\begin{itemize}
	\item $\PosInv$: objects are involutive posets and morphisms are monotone maps preserving the involution.
	\item $\SupInv$: objects are involutive suplattices and morphisms preserve all joins and involution.
\end{itemize}

There is an obviously defined forgetful functor $G\colon\SupInv\to\PosInv$.
For an involutive  poset $P$, we write $F(P)$ for the set of all orthoclosed lower sets of $P$.
It is easy to see that $I\mapsto I^\perp$ is an involution on $F(P)$. Moreover,
for every element $x$ of $P$, $(x^\downarrow)\perp=(x')^\downarrow$.
Thus, there is a $\PosInv$ morphism $\downarrow\colon P\to GF(P)$ that takes every
element $x$ to its principal downset
$$
\downarrow(x)=\{y\in P:y\leq x\}
$$
\begin{theorem}
Let $A$ be an object of $\PosInv$ and $B$ an object of $\SupInv$.
For every morphism $f\colon A\to G(B)$ there if a unique $\widehat f\colon F(A)\to B$ 
such that
$$
\xymatrix{
GF(A)
	\ar[r]^{G(\widehat f)}
&
G(B)
\\
A
	\ar[u]^{\downarrow}
	\ar[ru]_{f}
}
$$
commutes.

Let us prove that $\widehat f$ is unique.
\end{theorem}





\end{document}
